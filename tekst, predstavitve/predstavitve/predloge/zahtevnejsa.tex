% $Header: /cvsroot/latex-beamer/latex-beamer/solutions/generic-talks/generic-ornate-15min-45min.en.tex,v 1.4 2004/10/07 20:53:08 tantau Exp $

\documentclass{beamer}

% This file is a solution template for:

% - Giving a talk on some subject.
% - The talk is between 15min and 45min long.
% - Style is ornate.



% Copyright 2004 by Till Tantau <tantau@users.sourceforge.net>.
%
% In principle, this file can be redistributed and/or modified under
% the terms of the GNU Public License, version 2.
%
% However, this file is supposed to be a template to be modified
% for your own needs. For this reason, if you use this file as a
% template and not specifically distribute it as part of a another
% package/program, I grant the extra permission to freely copy and
% modify this file as you see fit and even to delete this copyright
% notice. 


\mode<presentation>
{
  \usetheme{Warsaw}
  % or ...

  \setbeamercovered{transparent}
  % or whatever (possibly just delete it)
}


\usepackage[english]{babel}
% or whatever

\usepackage[latin1]{inputenc}
% or whatever

\usepackage{times}
\usepackage[T1]{fontenc}
% Or whatever. Note that the encoding and the font should match. If T1
% does not look nice, try deleting the line with the fontenc.

\newcommand{\bx}{\mathbf{x}}
\DeclareMathOperator{\Gl}{Gl}
\DeclareMathOperator{\adj}{adj}
%\newcommand{\Gl}{{\mathrm {GL}}\,}
\newtheorem{proposition}[theorem]{Proposition}


\title[] % (optional, use only with long paper titles)
{Prenova \v studijskih programov na Fakulteti za matematiko in fiziko Univerze v Ljubljani - Matematika}

%\subtitle
%{Presentation Subtitle} % (optional)

\author[T. Ko\v sir] % (optional, use only with lots of authors)
{Toma\v z Ko\v sir%\inst{1} \and S.~Another\inst{2}
}
% - Use the \inst{?} command only if the authors have different
%   affiliation.

%\institute[] % (optional, but mostly needed)
%{
%  \inst{1}%
%  Department of Mathematics\\
%  University of Ljubljana\\
%  Slovenia
%  \and
%  \inst{2}%
%  Department of Theoretical Philosophy\\
%  University of Elsewhere
%}
% - Use the \inst command only if there are several affiliations.
% - Keep it simple, no one is interested in your street address.

\date[9. november 2007] % (optional)
{Ob\v cni zbor DMFA Slovenije, Pod\v cetrtek\\ 9. november  2007}

\subject{Talks}
% This is only inserted into the PDF information catalog. Can be left
% out. 



% If you have a file called "university-logo-filename.xxx", where xxx
% is a graphic format that can be processed by latex or pdflatex,
% resp., then you can add a logo as follows:

% \pgfdeclareimage[height=0.5cm]{university-logo}{university-logo-filename}
% \logo{\pgfuseimage{university-logo}}



% Delete this, if you do not want the table of contents to pop up at
% the beginning of each subsection:
%\AtBeginSubsection[]
%{
%  \begin{frame}<beamer>
%    \frametitle{Outline}
%    \tableofcontents[currentsection,currentsubsection]
%  \end{frame}
%}


% If you wish to uncover everything in a step-wise fashion, uncomment
% the following command: 

%\beamerdefaultoverlayspecification{<+->}


\begin{document}

\begin{frame}
  \titlepage
\end{frame}

%\begin{frame}
%  \frametitle{Outline}
%  \tableofcontents
%  % You might wish to add the option [pausesections]
%\end{frame}


% Since this a solution template for a generic talk, very little can
% be said about how it should be structured. However, the talk length
% of between 15min and 45min and the theme suggest that you stick to
% the following rules:  

% - Exactly two or three sections (other than the summary).
% - At *most* three subsections per section.
% - Talk about 30s to 2min per frame. So there should be between about
%   15 and 30 frames, all told.

%\section{Determinantal Representations}

%\subsection[Definition]{Definition}

\begin{frame}
  \frametitle{Prenovljeni \v studijski programi na OM FMF}
%  \framesubtitle{Subtitles are optional.}
  % - A title should summarize the slide in an understandable fashion
  %   for anyone how does not follow everything on the slide itself.

\begin{columns}
\begin{column}{5.7cm}
Na Oddelku za matematiko FMF v Ljubljani smo pripravili univerzitetne \v studijske programe:
\begin{itemize}

\item \alert{Matematika} 1. in 2. stopnje (3+2),\pause
\item \alert{Finan\v cna matematika} 1. in 2. stopnje (3+2),\pause
\item \alert{Pedago\v ska matematika} 1. in 2. stopnje (4+1),\pause
\end{itemize}
in visoko\v solski strokovni \v studijski program:
\begin{itemize}
\item \alert{Prakti\v cna matematika} (3-letni program).
\end{itemize}
\end{column}
\begin{column}{5.3cm}
\begin{figure}

\only<3-4>{
\includegraphics[width=5.3cm]{vhod0} }

\only<1-2>
{\includegraphics[width=5.8cm]{napis2} }
\end{figure}
\end{column}
\end{columns}

\end{frame}

\begin{frame}\frametitle{Pri\v cetek izvajanja}
\begin{itemize}
\item<1-> Programa \alert{Matematika} in \alert{Finan\v cna matematika} 1. stopnje smo pri\v celi izvajati letos. \pause
\item<2-> Program \alert{Prakti\v cna matematika} bomo predvidoma za\v celi izvajati v \v studijskem letu 2008/09.\pause
\item<3> Program \alert{Pedago\v ska matematika} \v se \v caka na akreditacijo. Malo verjetno, da se pri\v cne 
izvajati v 2008/09.
\end{itemize}

\begin{columns}
\begin{column}{5.4cm}
\only<1-2>{
\begin{figure}
\includegraphics[width=5.4cm]{vhod3} 
\end{figure}}

\only<3>{
\begin{figure}
\includegraphics[width=5.4cm]{vhod4} 
\end{figure}}

\end{column}
\begin{column}{5.4cm}

\only<2-3>{
\begin{figure}
\includegraphics[width=5.4cm]{ucenje}
\end{figure}}

\only<1>{
\begin{figure}
\includegraphics[width=5.4cm]{predavalnica}
\end{figure}}
\end{column}
\end{columns}

\end{frame}

\begin{frame}\frametitle{Osnovno o programih}

\begin{itemize}
\item Program \alert{Matematika} (3+2) je nadomestil Uporabno smer in Teoreti\v cno smer starega 
programa Matematika. \pause
\item Program \alert{Pedago\v ska matematika} bo nadomestil Pedago\v sko smer starega programa Matematika. 
Do njegovega sprejema se po starem izvajata \v se dve smeri
- Pedago\v ska smer in smer Ra\v cunalni\v stvo z matematiko.\pause
\item Program \alert{Finan\v cna matematika} je povsem nov \v studijski program.\pause
\item Visoko\v solski strokovni program ostaja en sam z enakim imenom \alert{Prakti\v cna matematika}.
\end{itemize}
\end{frame}

\begin{frame}
\frametitle{Kaj je novega?}
\begin{columns}
\begin{column}{5.4cm}
\only<1-2>{
\begin{figure}
\includegraphics[width=5.4cm]{Knjiznica} 
\end{figure}}
\only<3->{
\begin{figure}
\includegraphics[width=5.4cm]{avla2} 
\end{figure}}
\end{column}
\begin{column}{5.4cm}
\only<1>{
\begin{figure}
\includegraphics[width=5.4cm]{tabla2}
\end{figure}}

\only<2>{
\begin{figure}
\includegraphics[width=5.4cm]{zapiski}
\end{figure}}

\only<3->{
\begin{figure}
\includegraphics[width=5.4cm]{tabla}
\end{figure}}
\end{column}
\end{columns}

\vskip 0.45cm
{\bf \large{Ve\v cja izbirnost:}}
\begin{itemize}
\item V 3. letniku in na 2. stopnji lahko \v student
v veliki meri izbere predmete po svojih \v zeljah.\pause
\item Na 2. stopnji poleg matemati\v cnih predmetov obvezno izbere \v se nematemati\v cne predmete.\pause 
Tako lahko pridobi znanja, ki mu pomagajo pri iskanju zaposlitve.
\end{itemize}
\end{frame}

\begin{frame}\frametitle{Kaj je novega?}

\begin{columns}
\begin{column}{6.5cm}
{\bf \large{Povezava s prakso:}}

\begin{itemize}
\item Prakti\v cno usposabljanje je lahko del programa na 2. stopnji programa \alert{Matematika},\pause in je obvezno na 2. 
stopnji programa \alert{Finan\v cna matematika} in na programu \alert{Prakti\v cna matematika} (to zadnje je enako, kot doslej).\pause
\item Obvezna praksa pouka v obsegu 15 ECTS na programu \alert{Pedago\v ska matematika}.
\end{itemize}
\end{column}

\begin{column}{4.5cm}
\only<1->{
\begin{figure}
\includegraphics[width=4.5cm]{racunalnica} 
\end{figure}}

\end{column}
\end{columns}
\end{frame}

\begin{frame}
\frametitle{Kaj je novega?}
{\bf \large{Sproten \v studij:}}
\begin{itemize}
\item doma\v ce naloge, kvizi, tutorstvo, proseminar, projektne naloge\ldots.\pause
\item semestrski predmeti, kjer so bile te\v zave s prehodnostjo, drugje celoletni, saj obilica semstrskih predmetov ne
pospe\v suje prehodnosti po tujih izku\v snjah.\pause
\end{itemize}
\begin{columns}
\begin{column}{5.4cm}
\only<1->{
\begin{figure}
\includegraphics[width=5.4cm]{vega} 
\end{figure}}
\end{column}
\begin{column}{5.4cm}
\only<1->{
\begin{figure}
\includegraphics[width=5.4cm]{avlav}
\end{figure}}
\end{column}
\end{columns}
\end{frame}


\end{document}




\begin{columns}
\begin{column}{5.4cm}
\only<1-2>{
\begin{figure}
\includegraphics[width=5.4cm]{Knjiznica} 
\end{figure}}

\end{column}
\begin{column}{5.4cm}

\only<1-2>{
\begin{figure}
\includegraphics[width=5.4cm]{ucenje}
\end{figure}}

\end{column}
\end{columns}









\begin{frame}
  \frametitle{Druga orodja}
\begin{columns}
\begin{column}{5.7cm}
  \begin{itemize}
  \item numeri\v cne metode,
    \item iz matemati\v cne analize - diferencialne ena\v cbe, iskanje negibnih to\v ck\ldots
  \item iz kombinatorike in diskretne matematike,
  \item iz teorije iger, itd.
  \end{itemize}
\end{column}
\begin{column}{5.3cm}
\begin{figure}
\includegraphics[scale=0.4]{bloomberg} 
\end{figure}
\end{column}
\end{columns}
\end{frame}
