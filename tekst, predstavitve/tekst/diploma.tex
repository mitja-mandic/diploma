\documentclass[12pt,a4paper]{amsart}

% ukazi za delo s slovenscino -- izberi kodiranje, ki ti ustreza
\usepackage[utf8]{inputenc}
\usepackage[T1]{fontenc}
\usepackage[slovene]{babel}
\usepackage{lmodern} 
%\usepackage[T1]{fontenc}
%\usepackage[utf8]{inputenc}
\usepackage{amsmath,amssymb,amsfonts}
\usepackage{url}
%\usepackage[normalem]{ulem}
\usepackage[dvipsnames,usenames]{color}

% ne spreminjaj podatkov, ki vplivajo na obliko strani
\textwidth 15cm
\textheight 24cm
\oddsidemargin.5cm
\evensidemargin.5cm
\topmargin-5mm
\addtolength{\footskip}{10pt}
\pagestyle{plain}
\overfullrule=15pt % oznaci predlogo vrstico


% ukazi za matematicna okolja
\theoremstyle{definition} % tekst napisan pokoncno
\newtheorem{definicija}{Definicija}[section]
\newtheorem{primer}[definicija]{Primer}
\newtheorem{opomba}[definicija]{Opomba}

\renewcommand\endprimer{\hfill$\diamondsuit$}


\theoremstyle{plain} % tekst napisan posevno
\newtheorem{lema}[definicija]{Lema}
\newtheorem{izrek}[definicija]{Izrek}
\newtheorem{trditev}[definicija]{Trditev}
\newtheorem{posledica}[definicija]{Posledica}


% za stevilske mnozice uporabi naslednje simbole
\newcommand{\R}{\mathbb R}
\newcommand{\N}{\mathbb N}
\newcommand{\Z}{\mathbb Z}
\newcommand{\C}{\mathbb C}
\newcommand{\Q}{\mathbb Q}


% ukaz za slovarsko geslo
\newlength{\odstavek}
\setlength{\odstavek}{\parindent}
\newcommand{\geslo}[2]{\noindent\textbf{#1}\hspace*{3mm}\hangindent=\parindent\hangafter=1 #2}


% naslednje ukaze ustrezno popravi
\newcommand{\program}{Finančna matematika} % ime studijskega programa: Matematika/Finan"cna matematika
\newcommand{\imeavtorja}{Mitja Mandić} % ime avtorja
\newcommand{\imementorja}{izred.~prof.~dr. Jaka Smrekar} % akademski naziv in ime mentorja
\newcommand{\naslovdela}{Iterativne numerične metode v posplošenih linearnih modelih}
\newcommand{\letnica}{2021} %letnica diplome


% vstavi svoje definicije ...




\begin{document}

% od tod do povzetka ne spreminjaj nicesar
\thispagestyle{empty}
\noindent{\large
UNIVERZA V LJUBLJANI\\[1mm]
FAKULTETA ZA MATEMATIKO IN FIZIKO\\[5mm]
\program\ -- 1.~stopnja}
\vfill

\begin{center}{\large
\imeavtorja\\[2mm]
{\bf \naslovdela}\\[10mm]
Delo diplomskega seminarja\\[1cm]
Mentor: \imementorja}
\end{center}
\vfill

\noindent{\large
Ljubljana, \letnica}
\pagebreak

\thispagestyle{empty}
\tableofcontents
\pagebreak

\thispagestyle{empty}
\begin{center}
{\bf \naslovdela}\\[3mm]

{\sc Povzetek}
\end{center}
% tekst povzetka v slovenscini
V povzetku na kratko opi"si vsebinske rezultate dela. Sem ne sodi razlaga organizacije dela -- v katerem poglavju/razdelku je kaj, pa"c pa le opis vsebine.
\vfill
\begin{center}
{\bf Angle"ski naslov dela}\\[3mm] % prevod slovenskega naslova dela
{\sc Abstract}
\end{center}
% tekst povzetka v anglescini
Prevod zgornjega povzetka v angle"s"cino.

\vfill\noindent
{\bf Math. Subj. Class. (2010):} navedi vsaj eno klasifikacijsko oznako -- dostopne so na \url{www.ams.org/mathscinet/msc/msc2010.html}  \\[1mm]
{\bf Klju"cne besede:} navedi nekaj klju"cnih pojmov, ki nastopajo v delu  \\[1mm]
{\bf Keywords:} angle"ski prevod klju"cnih besed
\pagebreak



% tu se zacne besedilo seminarja
\section{Uvod}







% slovar
\section*{Slovar strokovnih izrazov}

%\geslo{}{}
%
%\geslo{}{}
%


% seznam uporabljene literature
%\begin{thebibliography}{99}

%\bibitem{}

%end{thebibliography}

\nocite{*}
\bibliographystyle{plain}
\bibliography{literatura}


\end{document}

