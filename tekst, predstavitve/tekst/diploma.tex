\documentclass[12pt,a4paper]{amsart}

% ukazi za delo s slovenscino -- izberi kodiranje, ki ti ustreza
\usepackage[utf8]{inputenc}
\usepackage[T1]{fontenc}
\usepackage[slovene]{babel}
\usepackage{lmodern} 
%\usepackage[T1]{fontenc}
%\usepackage[utf8]{inputenc}
\usepackage{amsmath,amssymb,amsfonts}
\usepackage{url}
%\usepackage[normalem]{ulem}
\usepackage[dvipsnames,usenames]{color}
\usepackage{eurosym}
\usepackage{mathtools}
% ne spreminjaj podatkov, ki vplivajo na obliko strani
\textwidth 15cm
\textheight 24cm
\oddsidemargin.5cm
\evensidemargin.5cm
\topmargin-5mm
\addtolength{\footskip}{10pt}
\pagestyle{plain}
\overfullrule=15pt % oznaci predlogo vrstico


% ukazi za matematicna okolja
\theoremstyle{definition} % tekst napisan pokoncno
\newtheorem{definicija}{Definicija}[section]
\newtheorem{primer}[definicija]{Primer}
\newtheorem{opomba}[definicija]{Opomba}

\renewcommand\endprimer{\hfill$\diamondsuit$}


\theoremstyle{plain} % tekst napisan posevno
\newtheorem{lema}[definicija]{Lema}
\newtheorem{izrek}[definicija]{Izrek}
\newtheorem{trditev}[definicija]{Trditev}
\newtheorem{posledica}[definicija]{Posledica}


% za stevilske mnozice uporabi naslednje simbole
\newcommand{\R}{\mathbb R}
\newcommand{\N}{\mathbb N}
\newcommand{\Z}{\mathbb Z}
\newcommand{\C}{\mathbb C}
\newcommand{\Q}{\mathbb Q}


% ukaz za slovarsko geslo
\newlength{\odstavek}
\setlength{\odstavek}{\parindent}
\newcommand{\geslo}[2]{\noindent\textbf{#1}\hspace*{3mm}\hangindent=\parindent\hangafter=1 #2}


% naslednje ukaze ustrezno popravi
\newcommand{\program}{Finančna matematika} % ime studijskega programa: Matematika/Finan"cna matematika
\newcommand{\imeavtorja}{Mitja Mandić} % ime avtorja
\newcommand{\imementorja}{izred.~prof.~dr. Jaka Smrekar} % akademski naziv in ime mentorja
\newcommand{\naslovdela}{Iterativne numerične metode v posplošenih linearnih modelih}
\newcommand{\letnica}{2021} %letnica diplome


% vstavi svoje definicije ...




\begin{document}

% od tod do povzetka ne spreminjaj nicesar
\thispagestyle{empty}
\noindent{\large
UNIVERZA V LJUBLJANI\\[1mm]
FAKULTETA ZA MATEMATIKO IN FIZIKO\\[5mm]
\program\ -- 1.~stopnja}
\vfill

\begin{center}{\large
\imeavtorja\\[2mm]
{\bf \naslovdela}\\[10mm]
Delo diplomskega seminarja\\[1cm]
Mentor: \imementorja}
\end{center}
\vfill

\noindent{\large
Ljubljana, \letnica}
\pagebreak

\thispagestyle{empty}
\tableofcontents
\pagebreak

\thispagestyle{empty}
\begin{center}
{\bf \naslovdela}\\[3mm]

{\sc Povzetek}
\end{center}
% tekst povzetka v slovenscini
V povzetku na kratko opi"si vsebinske rezultate dela. Sem ne sodi razlaga organizacije dela -- v katerem poglavju/razdelku je kaj, pa"c pa le opis vsebine.
\vfill
\begin{center}
{\bf Iterative numerical methods in generalized linear models}\\[3mm] % prevod slovenskega naslova dela
{\sc Abstract}
\end{center}
% tekst povzetka v anglescini


\vfill\noindent
{\bf Math. Subj. Class. (2010):} navedi vsaj eno klasifikacijsko oznako -- dostopne so na \url{www.ams.org/mathscinet/msc/msc2010.html}  \\[1mm]
{\bf Klju"cne besede:} navedi nekaj klju"cnih pojmov, ki nastopajo v delu  \\[1mm]
{\bf Keywords:} angle"ski prevod klju"cnih besed
\pagebreak



% tu se zacne besedilo seminarja
\section{Uvod}
Posplošeni linearni modeli so ključen del statistične analize. Pomagajo nam bolje razumeti relacije med rezultati meritev in s temi izsledki
predvideti trende v prihodnosti. Modeli morajo biti kar se da enostavni, ampak hkrati zagotavljati določeno natančnost. Vsa teorija nam pa v praksi
ne koristi, če konkretnih številk ne znamo izračunati.

V dobi ogromne množice podatkov je računska učinkovitost ključni del obdelave. Tu nam pomagajo numerične metode.

V delu bom najprej predstavil posplošene linearne modele in najpriljubljenejše numerične metode, ki se uporabljajo za njihovo obdelavo. 
Teorijo bom osvetlil tudi s praktičnimi primeri.

\section{Posplošeni linearni modeli}

\subsection{Sestavni deli posplošenega linearnega modela}
Vsak posplošeni linearni model sestavljajo trije deli: \textit{slučajni del} je slučajna spremenljivka $Y$ in njena porazdelitev, 
\textit{sistematični del} predstavlja relacijo med pojasnjevalnimi spremenljivkami, \textit{povezovalna funkcija} pa transformira $Y$, da se ta
bolje prilega podatkom.

\subsubsection{Slučajni del}
\textit{Slučajni del} privzame porazdelitev slučajnega vektorja $Y$, pri čemer privzemamo tudi neodvisnost komponent. Porazdelitev $Y$
privzemamo odvisno od podatkov; mnogokrat je ,,binarna``, torej ima dve možni vrednosti - ,,uspeh`` ali ,,neuspeh``. Splošneje je lahko izid tudi 
število uspehov v fiksnem številu poskusov. V takih primerih privzamemo binomsko porazdelitev. $Y$ nam lahko meri tudi števne podatke, naprimer 
koliko zabav je obiskal študent v preteklem mesecu. Seveda pa lahko $Y$ predstavlja tudi zvezne podatke,
v tem primeru lahko privzamemo normalno porazdelitev (ali pa kakšno drugo zvezno porazdelitev).

\subsubsection{Sistematični del} 
\textit{Sistematična komponenta} posplošenega linearnega modela poda relacije med pojasnjevalnimi spremenljivkami $x_{i,j}$. Te nastopajo 
linearno, torej je sistematični del enak
\[
\beta_{0} + x_{i,1}\beta_{1} + x_{i,2}\beta_{2} + \ldots + x_{i,p}\beta_{p}
\]


\subsubsection{Povezovalna funkcija}
Tretji del posplošenega linearnega modela je \textit{povezovalna funkcija}, ta nam poda funkcijo $g(\cdot)$ med slučajno komponento
in sistematičnim delom. Če označimo $\mu = E(Y)$, je
\[
    g(\mu) = \beta_{0} + x_{i,1}\beta_{1} + x_{i,2}\beta_{2} + \ldots + x_{i,p}\beta_{p}
\]
Najenostavnejša taka funkcija je kar identiteta, torej $g(\mu) = \mu$. Ta nam torej da linearno povezavo med pojasnjevalnimi spremenljivkami 
in pričakovano vrednostjo naših slučajnih spremenljivki. To je ena od oblik regresije za zvezne podatke.

Mnogokrat pa linearna relacija ni primerna - fiksna sprememba pojasnjevalnih spremenljivk ima lahko večji vpliv, če je pričakovana vrednost 
bližje 0, kot če je bližje 1. Recimo, da je $\pi$ verjetnost, da bo oseba kupila nov avto, ko je njen dohodek enak $x$. Sprememba v dohodku
za 10.000\euro~ima manjši vpliv, če je dohodek 1.000.000\euro,~kot če je 50.000\euro.

Takrat je smiselno uporabiti kakšno drugo povezovalno funkcijo, ki dopušča tudi nelinearne kombinacije pojasnjevalnih
spremenljivk. Naprimer, $g(\mu) = \log(\mu)$ modelira
pričakovano vrednost logaritma. Smiselno jo je uporabiti, če pričakovana vrednost ne more zavzeti negativnih vrednosti. Takemu modelu rečemo
\textit{log-linearen} model.

Spet druga povezovalna funkcija je $\mathrm{logit}(\mu) = \log(\frac{\mu}{1-\mu})$, ki nam modelira logaritem deležev - smiselno jo je uporabiti, 
ko $\mu$ ne zavzame vrednosti izven $(0,1)$, torej ko imamo opravka z verjetnostmi. Takemu modelu rečemo logistični model.

%\subsection{Primeri posplošenih linearnih modelov}

\subsection{Linearna regresija}

Linearna regresija je najenostavnejši primer posplošenega linearnega modela. Enostavno jo lahko zapišemo kot:
$
    Y = X \beta + \varepsilon
$
kjer je $Y$ proučevan slučajni vektor, $X$ je matrika pojasnjevalnih slučajnih spremenljivk, $\beta$ je vektor koeficientov, ki jih želimo oceniti,
$\varepsilon$ pa slučajna spremenljivka, ki predstavlja napako - pri računanju, meritvah \ldots. Privzemimo, da je $E(\varepsilon) = 0$. Iz tega sledi
$\mu = E(Y) = X\beta$. Model torej pričakovano vrednost slučajne spremenljivke predstavi kot linearno funkcijo pojasnjevalnih spremenljivk.
Parametre $\beta$ ocenimo z metodo najmanjših kvadratov in ob predpostavki polnega ranga za matriko $X$ dobimo $\hat{\beta} =  (X^\top X)^{-1}
X^\top Y$. 


\subsection{Poissonova regresija}

\subsection{Logistična regresija}
Logistična regresija se uporablja za določanje deležev oziroma računanje verjetnosti. V poštev pride, ko imamo odgovore tipa uspeh-neuspeh oziroma
govorimo o prisotnosti ali odsotnosti neke lastnosti.

Spomnimo se Binomske porazdelitve $Y_{i} \sim B(n_{i}, p_{i})$. Ta pravi, da je 
\[
    P(Y_{i} = y_{i}) = {n_{i} \choose y_{i}} p_{i}^{y_{i}}(1 - p_{i}) ^{n_{i} - y_{i}}
\]
Pričakovana vrednost in varianca sta odvisni le od $p_{i}$, in sta enaki $E(Y_{i}) = p_{i} \text{in} Var(Y_{i}) = p_{i}(1 - p_{i})$.
Poglejmo si sedaj podrobneje \textit{logit} transformacijo. Če se spomnemo, želimo določiti verjetnost nekega dogodka pri danih podatkih. Ob uporabi
identitente transformacije se nam kaj hitro lahko zgodi, da za posamezne verjetnosti dobimo vrednosti izven intervala $[0,1]$. Ta problem bomo rešili v 
dveh korakih.

Najprej uvedimo 
\[ 
    \mathrm{delež}_{i} = \frac{p_{i}}{1 - p_{i}} %poglej kako se tle da vejico
\]
kjer se premaknemo iz verjetnosti v \textit{deleže} -- verjetnost dogodka proti verjetnosti, da se ne bo zgodil. Če je $p_{i}$ enak $\frac{1}{2}$, 
bo delež enak 1. Vidimo, da so deleži vedno pozitivni in niso omejeni navzgor

V naslednjem koraku pa poglejmo logaritem deležev ali logit verjetnosti
\[
    \eta_{i} = \mathrm{logit}(p_{i}) = \log \frac{p_{i}}{1 - p_{i}}
\]
s tem pa si odstranimo tudi omejitev navzdol. Opazimo še, da če je $p_{i} = \frac{1}{2}$, je delež enak 1 in je logaritem 0. Kot funkcija p, je logit
strogo naraščajoča, torej imamo inverz. Običajno ga imenujemo \textit{antilogit}, izrazimo ga z:
\[
    p_{i} = \mathrm{logit}(\eta_{i}) = \frac{\exp{\eta_{i}}}{1+\exp{\eta_{i}}}
\]
Vse skupaj nam da \textit{logistični model}, ki za slučajni del vzame binomsko porazdelitev. %VPRAŠAJ KAJ JE Z NAPAKO

\subsubsection{Ocenjevanje parametrov}
Imamo binomske slučajne spremenljivke in imamo povezovalno funkcijo, $\mathrm{}{logit p_{i}} = X\beta$, kjer so $\beta$ neznani parametri.
V naslednjem razdelku si bomo ogledali kako zanje izpeljemo enačbe verjetja, ki jih nato uporabimo v numeričnih algoritmih.

Kot v vsakem posplošenem linearnem modelu tudi v tem predpostavimo neodvisnost komponent slučajnega vektorja $\mathrm{Y}$ zato 
\begin{align*}
    P(Y = \vec{y}) &= \prod_{i=1}^{n} P(Y_{i} = y_{i}) \\
                    &=\prod_{i=1}^{n} {n_{i} \choose y_{i}} p_{i}^{y_{i}}(1 - p_{i})^{n_{i} - y_{i}}
\end{align*}
Naprej si oglejmo logaritemsko funkcijo verjetja. V nadaljnem računanju bom izpuščal binomski simbol na začetku - je samo konstanta, ki na
končen rezultat nima vpliva. Po prejšnjih oznakah je torej
\begin{align}
    \ell(p_{i}) &= \log\{\prod_{i=1}^{n} p_{i}^{y_{i}}(1 - p_{i})^{n_{i} - y_{i}} \} \nonumber  \\
        &= \sum_{i=1}^{n}\{y_{i}\log{p_{i}} + (n_{i} - y_{i})\log(1 - p_{i})\} \nonumber \\
        &= \sum_{i=1}^{n}\{n_{i}\log{1-p_{i}}  + y_{i}\log{\left(\frac{p_{i}}{1-p_{i}}\right)}\}
\end{align}

Po predpostavki logističnega modela je 
\[
   \mathrm{logit}(p_{i}) = \log\left( \frac{p_{i}}{1-p_{i}}  \right) = \beta_{0} + x_{i1}\beta_{1} + \ldots + x_{ir}\beta_{r} = x_{i}^\top \beta 
\]


\subsection{Probit regresija}

\subsection{Ocenjevanje parametrov}

\section{Numerične metode}

\subsection{Newton -- Raphsonova metoda}

\subsection{Fisher-scoring algoritem}

\section{Primeri}
\subsection{Ocenjevanje parametrov v logističnem modelu}
\subsection{Ocenjevanje parametrov v probit modelu}

% slovar
\section*{Slovar strokovnih izrazov}

%\geslo{}{}
%
%\geslo{}{}
%


% seznam uporabljene literature
%\begin{thebibliography}{99}

%\bibitem{}

%end{thebibliography}

\nocite{*}
\bibliographystyle{plain}
\bibliography{literatura}


\end{document}

