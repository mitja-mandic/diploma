\documentclass[14pt]{beamer}
\usepackage[T1]{fontenc}
\usepackage[utf8]{inputenc}
\usepackage[slovene]{babel}
\usepackage{pgfpages} % privat zapiski
\usepackage{amsmath} % pravilen izpis v "math mode"
\usepackage{hyperref}
\usepackage{pgfplots}
\usepackage{tikz}
\pgfplotsset{compat=1.8}
\hypersetup{hidelinks}
%\usetheme{Bergen}
\usecolortheme{seahorse}

\usepackage{graphicx}% http://ctan.org/pkg/graphicx
\usepackage{booktabs}% http://ctan.org/pkg/booktabs

\usepackage{palatino}
\usefonttheme{serif}

\setbeamertemplate{navigation symbols}{} % izklop navigacije
\setbeamertemplate{footline}[frame number]{} % oštevilčenje
\setbeamertemplate{note page}{\pagecolor{yellow!5}\insertnote}
\setbeamertemplate{itemize items}[circle]

\begin{document}
    \title[Diplomski seminar]{Iterativne numerične metode v posplošenih linearnih modelih}
    \author[Mitek]{Mitja Mandić \\ \small Mentor: izred. prof. dr. Jaka Smrekar}
    \date{20. november 2020} 

\begin{frame}
    \titlepage
\end{frame}

\begin{frame} \frametitle{Posplošeni linearni modeli}
    \begin{itemize}
    
        \item Slučajni del, sistematični del, povezovalna funkcija
        \pause
        \item Linearna regresija: 
            $$ Y = x ^ T\beta + \epsilon $$
        \item Problem - ni najboljša. Rešitev? Transformacija Y
    \end{itemize}
\end{frame}

\begin{frame} \frametitle{Logistični model}
    \begin{itemize}
    \item Za binarne podatke $\rightarrow$ binomska porazdelitev
    \pause
    \item $ \text{logit}(p_{i}) = \log (\frac{p_{i}}{1 - p_{i}}) = x ^ T\beta $ %\rightarrow p_{i} = \frac{ e^{x^T\beta}}{1 + e^{x^T\beta}} $

    \item Sedaj je treba izračunati $\beta$ $\rightarrow$ Numerične metode

\end{itemize}

%\begin{figure}    treba še naštudirat da bo graf prou
%\centering
%    \begin{tikzpicture}[transform canvas={scale=0.5}]
%        
%        \begin{axis}%
%        [
%            grid=major,     
%            xmin=-10,
%            xmax=10,
%            axis x line=bottom,
%            ytick={0,.5,1},
%            ymax=1,
%            axis y line=middle,
%        ]
%            \addplot%
%            [
%                blue,%
%                mark=none,
%                samples=100,
%                domain=-6:6,
%            ]
%            (x,{1/(1+exp(-x))});
%        \end{axis}
%    \end{tikzpicture}
%\end{figure}
\end{frame}


\begin{frame}\frametitle{Numerične metode}
    \begin{itemize}
        \item Za ocenjevanje parametrov $\beta$ običajno rešujemo sistem enačb največjega verjetja
        \pause
        %\item v splošnem ni eksplicitno rešljiv
        \item Newtonova metoda še vedno zelo aktualna:
        $$ x_{i+1} = x_{i} - \frac{f'(x_{i})}{f''(x_{i})} $$
        %\pause
        
        \item Izboljšava: Fisher-scoring  
    \end{itemize}
\end{frame}

\begin{frame}{Fisher Scoring}

    $$ \beta_{i+1} = \beta_{i} + \frac{\dot{l(\beta_{i})}}{E(\ddot{l(\beta_{i}))}} $$
    
    \begin{itemize}
        \item za logistično regresijo sovpadata z Newtonovo metodo
        \item Informacijska matrika je pozitivno definitna $\rightarrow$ imamo naraščajoč algoritem
    \end{itemize}

\end{frame}

\begin{frame}{Kaj sem že naredil}
    Mal rezultati pa tko
\end{frame}

\begin{frame}{Kaj še bom naredil}
    Mal bl na splošno še raziskal kako pa kaj
\end{frame}
\end{document}