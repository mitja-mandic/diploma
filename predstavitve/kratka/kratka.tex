\documentclass[14pt]{beamer}
\usepackage[T1]{fontenc}
\usepackage[utf8]{inputenc}
\usepackage[slovene]{babel}
\usepackage{pgfpages} % privat zapiski
\usepackage{amsmath} % pravilen izpis v "math mode"
\usepackage{hyperref}
\hypersetup{hidelinks}
%\usetheme{Bergen}
\usecolortheme{seahorse}

\usepackage{graphicx}% http://ctan.org/pkg/graphicx
\usepackage{booktabs}% http://ctan.org/pkg/booktabs

\usepackage{palatino}
\usefonttheme{serif}

\setbeamertemplate{navigation symbols}{} % izklop navigacije
\setbeamertemplate{footline}[frame number]{} % oštevilčenje
\setbeamertemplate{note page}{\pagecolor{yellow!5}\insertnote}
\setbeamertemplate{itemize items}[circle]

\begin{document}
    \title[Diplomski seminar]{Iterativne numerične metode v posplošenih linearnih modelih}
    \author[Mitek]{Mitja Mandić \\ \small Mentor: izred. prof. dr. Jaka Smrekar}
    \date{20. november 2020} 
\begin{frame}
    \titlepage
\end{frame}

\begin{frame} \frametitle{Statistični modeli}
    Neki povem na splošno o modelih
\end{frame}

\begin{frame} \frametitle{Statistični modeli - primer}
    Mal vse skup razložim Logit, zakaj pa kako, mogoče kaka enačbica pade
\end{frame}

\begin{frame}\frametitle{Numerične metode}
    Zakaj iterativne? Katere sploh so aktualne?
\end{frame}

\begin{frame}{Kaj sem že naredil}
    Mal rezultati pa tko
\end{frame}

\begin{frame}{Kaj še bom naredil}
    Mal bl na splošno še raziskal kako pa kaj
\end{frame}
\end{document}